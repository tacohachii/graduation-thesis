\chapter{重力崩壊型超新星爆発}
\label{chap:supernova}

この章では、重力崩壊後から爆発を起こすまでの様子を3つのフェーズに分けて説明していく.最初のフェーズはコア反跳によって生成された衝撃波が停滞している状態で,これを「爆発前段階」と呼ぶ.次の段階は衝撃の停滞がニュートリノ加熱によって解除され徐々に広がっている段階で,これを「爆発段階(Phase1)」と呼ぶ.最後のフェーズは衝撃波のエネルギーが落ちてくる物質を外に全て跳ね返すほど大きくなる状態で,これを「爆発段階(Phase2)」と呼ぶ.この3つのフェーズについて以下で見ていく.

\section{爆発前段階}

ここでは停滞した衝撃波が復活するかどうかを考える.衝撃波の外側から中心に向かって降着してくる物質はゲイン領域を通過する時にニュートリノからエネルギーを受け取るが,その受け取ったエネルギーが重力的な結合エネルギーを上回ることができると衝撃波が復活するようになる.ここでは,ゲイン領域内の降着物質に注目する.
\subsection{降着物質の降着率}

降着物質は重力崩壊直前星における中心から位置$r$にある質量シェル(密度$\rho$)が落ちてきたものである.この時,物質は自由落下で落ちてきていると仮定して,落ちてくるまでのタイムスケールは自由落下のタイムスケールの定数倍となる.したがって,ある質量シェルが中心まで落ちてくる時間は
$$
t=\sqrt{\frac{\pi}{4G\bar{\rho}}}\eqno{(2.1)}
$$
と表せる.ここで,$\bar{\rho}$は重力崩壊前の質量シェルにおける半径$r$の位置での平均密度である.以下の式で表される.
$$
\bar{\rho}=\frac{M}{\frac{4}{3}\pi r^3}\eqno{(2.2)}
$$
結果として,生じる質量降着率は,次のように表される.
$$
\dot{M}=\frac{2M}{t}\frac{\rho}{\bar{\rho}-\rho}\eqno{(2.3)}
$$

\subsection{ゲイン半径}

ゲイン半径$r_g$では,降着物質がその重力的なポテンシャルエネルギーの約半分を失い,ゲイン半径での光度$\frac{GM\dot{M}}{2r_g}$として失う.このことから,
$$
M^4r_g^2\propto\frac{GM\dot{M}}{2r_g}\eqno{(2.4)}
$$
つまり,
$$
r_g\propto\frac{\dot{M}^{1/3}}{M}\eqno{(2.5)}
$$
と表される.しかし,これは明らかに$\dot{M}$が小さい場合におかしな値になるので,下限値として中性子星の半径$r_0$を補完する.すると,ゲイン半径は以下のように表せる.
$$
r_g=\sqrt[3]{r_1^3\left(\frac{\dot{M}}{M\sun}\right)\left(\frac{M}{M\sun}\right)^{-3}+r_0^3}\eqno{(2.6)}
$$
計算では,$r_0=12\rm{km}$,$r_1=120\rm{km}$を用いる.
\subsection{衝撃波半径}

ゲイン半径が決定すると,衝撃波半径を決定することができる.ゲイン半径では,単位質量あたりのニュートリノの加熱速度と冷却速度が釣り合っている.そのことを考えると,
$$
T^6_g\propto\frac{L_{\nu}E_{\nu}^2}{r_g^2}\eqno{(2.7)}
$$
ここで,$P_g\propto T_g^4$であるから,
$$
P^{3/2}_g\propto\frac{L_{\nu}E_{\nu}^2}{r_g^2}\eqno{(2.8)}
$$
となる.

ここで,衝撃波の前後での密度の比を$\beta=\frac{\rho_{pre}}{\rho_{sh}}$とする.そして,衝撃波直前の速度$v_{pre}$が自由落下時の速度になることから
$$
v_{pre}=\sqrt{\frac{2GM}{r_{sh}}}\eqno{(2.9)}
$$
となり,これを用いて衝撃波直前の密度が
$$
\rho_{pre}=\frac{\dot{M}}{4\pi r2v_{pre}}\eqno{(2.10)}
$$
として得られることから,衝撃波での圧力$P_{sh}$が次のように決まる.
$$
P_{sh}=\frac{\beta-1}{\beta}\rho_{pre}v_{pre}^2\eqno{(2.11)}
$$
この式と式(8)を合わせると,
$$
r_{sh}\propto\frac{(L_{\nu}E^2_{\nu})^{4/9}r_g^{16/9}}{\dot{M}^{2/3}M^{1/3}}\propto\frac{L_{\nu}^{4/9}M^{5/9}r_g^{16/9}}{\dot{M}^{2/3}}\eqno{(2.12)}
$$
となる.ここで多次元での衝撃波の復活で理論的に推定された値の標準値$\alpha_{turb}\sim1.18$を用いて
$$
r_{sh}=\alpha_{turb}\times0.55\rm{km}\times\left(\frac{L_{\nu}}{10^{52}\rm{erg s^{-1}}}\right)^{4/9}\times(\alpha^3)^{4/9}\times\left(\frac{M}{M\sun}\right)^{5/9}\times\left(\frac{r_g}{10\rm{km}}\right)^{16/9}\times\left(\frac{M}{M\sun}\right)^{-2/3}\eqno{(2.13)}
$$
\subsection{ニュートリノ光度}

衝撃半径を計算する時にも必要なニュートリノ光度$L_{\nu}$について説明する.衝撃波半径に向かってニュートリノが飛び出してくるのには2種類の起源がある.1つは降着物質の重力的な結合エネルギーが変換されニュートリノが出てくる事によるニュートリノ光度$L_{acc}$で,もう1つは中性子星が時間経過によって冷やされることで放出されるニュートリノ光度$L_{diff}$である.

まず,降着成分によるニュートリノ光度$L_{acc}$であるが,これは,パラメータ$\zeta$を用いて,
$$
L_{acc}=\zeta\frac{GM\dot{M}}{r_g}\eqno{(2.14)}
$$
と表される.計算では,パラメータ$\zeta=0.71$を用いた.

次に,中性子星によるニュートリノ光度$L_{diff}$であるが,これは中性子星の結合エネルギー$E_{bind}$と冷却時間$\tau_{cool}$から求めることができる.
$$
L_{diff}=0.3\times\frac{E_{bind}}{\tau_{cool}}e^{-t/\tau_{cool}}\eqno{(2.15)}
$$
ここで,中性子星の結合エネルギー$E_{bind}$と冷却時間$\tau_{cool}$については以下を用いる.
$$
E_bind=0.084\times\left(\frac{M}{M\sun}\right)^2M\sun c^2\eqno{(2.16)}
$$

$$
\tau_{cool}=\tau_{1.5}\times\left(\frac{M}{1.5M\sun}\right)^{5/3}\eqno{(2.17)}
$$

これらにより,ニュートリノ光度$L_{\nu}$は
$$
L_{\nu}=L_{acc}+L_{diff}\eqno{(2.18)}
$$
で求められる.ここまでで,ゲイン半径$r_g$と衝撃波半径$r_{sh}$を求めることができた.

\subsection{加熱条件}

衝撃波を通過した降着物質がゲイン半径で重力的束縛エネルギーを超えるだけのニュートリノ加熱を得ることができた時,衝撃波の停滞は終わり衝撃波は復活をする.降着物質がゲイン領域を通過する時間を$\tau_{adv}$,降着物質がニュートリノから重力的な結合エネルギーを超えるエネルギーを得るのにかかる時間を$\tau_{heat}$とすると,衝撃波が復活する条件は,
$$
\tau_{heat}<\tau_{adv}\eqno{(2.19)}
$$
となる.

降着物質がゲイン領域を通過する時間$\tau_{adv}$は第一原理シュミレーションにより係数が設定され,
$$
\tau_{adv}=\frac{\int^{r_{sh}}_{r_g}4\pi r^2\beta\rho_{pre}(r_{sh}/r)^3\rm{dr}}{\dot{M}}\sim18\rm{ms}\left(\frac{r_{sh}}{100\rm{km}}\right)^{3/2}\left(\frac{M}{M\sun}\right)^{-1/2}\ln{\frac{r_{sh}}{r_g}}\eqno{(2.20)}
$$
と表される.

降着物質がニュートリノから重力的な結合エネルギーを超えるエネルギーを得るのにかかる時間$\tau_{heat}$は,ニュートリノの平均加熱率$\dot{q}_{\nu}$と平均結合エネルギー$e_g$で表すとができる.
$$
\tau_{heat}=\frac{|e_g|}{\dot{q}_{\nu}}\eqno{(2.21)}
$$
ここで,ニュートリノの平均加熱率$\dot{q}_{\nu}$は,
$$
\dot{q}_{\nu}\propto\frac{L_{\nu}E_{\nu}^2}{r_g^2}\eqno{(2.22)}
$$
と表すことができる.平均結合エネルギー$e_g$については,球対称のベルヌーイの定理を用いることで推定できる.総エンタルピー$h$と運動エネルギー密度$\frac{v^2}{2}$と重力ポテンシャル$\Phi$が保存すると考えるので,
$$
h+\frac{v^2}{2}+\Phi=0\eqno{(2.23)}
$$
と書ける.これを降着物質に当てはめると,
$$
\left(\epsilon_{therm}+\epsilon_{diss}+\frac{P_{sh}}{\rho_{sh}}\right)+0-\frac{GM}{r_{sh}}\sim0\eqno{(2.24)}
$$
と表される.ここで,衝撃波直後の単位質量あたりのエネルギーを$\epsilon_{therm}$とし,そのうちの静止質量の寄与を$\epsilon_{diss}$としている.また,放射圧の影響が強いため,$P_{sh}/\rho_{sh}=\epsilon/3$とできる.これを解くと,
$$
\epsilon_{therm}=\frac{4}{3}\left(\frac{GM}{r_{sh}-\epsilon_{diss}}\right)\eqno{(2.25)}
$$
よって,静止質量の寄与が無い衝撃波通過後の結合エネルギー$|e_g|$は,
$$
|e_g|=|\epsilon_{therm}-\frac{GM}{r_{sh}}|=\frac{3}{4}\epsilon_{diss}+\frac{GM}{4max(r_{sh},r_g)}\eqno{(2.26)}
$$
と表される.ここで,重い原子核が落下中に完全に核子に分離されると想定すると$\epsilon_{diss}\sim8.8\rm{MeV}$とでき,計算でもこの値を用いた.

したがって,式(22)(23)より,加熱のタイムスケール$\tau_{heat}$は,
$$
\tau_{heat}=150\rm{ms}\times\left(\frac{|e_g|}{10^{19}\rm{erg g^{-1}}}\right)\times\left(\frac{r_g}{100\rm{km}}\right)^2\times\left(\frac{L_{\nu}}{10^{52}\rm{erg s^{-1}}}\right)^{-1}\times(\alpha^3)^{-1}\times\left(\frac{M}{M\sun}\right)^{-2}\eqno{(2.27)}
$$
となる.これで式(19)を評価することができる.式(19)を満たしたら次に説明する,爆発段階(Phase1)に進む.

\section{爆発段階(Phase1)}

この段階では,衝撃波が復活しているが衝撃波に達した降着物質が全て吹き飛ばされるわけではなく,まだ衝撃波を通過する降着物質が多いという状況を考えている.つまり,降着率$\dot{M}_{acc}$と外に出て行く質量率$\dot{M}_{out}$が以下の関係にあると考える.
$$
\dot{M}_{acc}\gg\dot{M}_{out}\eqno{(2.28)}
$$
これは衝撃波を通過した降着物質の速度$v_{post}$が脱出速度$v_{esc}$を超えていないためである.つまりこのフェーズの終わりの条件は,
$$
v_{post}>v_{esc}\eqno{(2.29)}
$$
となる時である.脱出速度$v_{esc}$は簡単に求まり,
$$
v_{esc}=\sqrt{\frac{2GM}{r}}\eqno{(2.30)}
$$
である.衝撃波を通過した降着物質の速度$v_{post}$は圧縮比$\beta_{expl}$を用いて,
$$
v_{post}=\frac{\beta_{expl}-1}{\beta_{expl}}v_{sh}\eqno{(2.31)}
$$
として表される.以下では,衝撃波の速度$v_{sh}$を求める.

\subsection{爆発のエネルギー}

衝撃波の速度$v_{sh}$は,衝撃波を通過してバラバラになった物質がニュートリの加熱によって外向きに進み,その降着物質が再結合した際に放出されるエネルギーに依存するものである.よって,そのエネルギーを爆発のエネルギー$E_{diag}$とし,先にこれを求めることにする.この爆発のエネルギー$E_{diag}$は累次的に足し合わせて行くことで最終的には重力崩壊型超新星爆発の爆発エネルギーとなる.

爆発のエネルギー$E_{diag}$は,
$$
\frac{\rm{d} E_{diag}}{\rm{d} M_{sh}}=\frac{(1-\alpha_{out})\epsilon_{rec}\eta_{acc}}{|e_g|}+\alpha_{out}(\epsilon_{bind}+\epsilon_{burn})\eqno{(2.32)}
$$
と表される.ここで,$1-\alpha_{out}$の部分が衝撃波を通過する割合を表しており,$\epsilon_{rec}$が再結合のエネルギーを表している.計算では再結合のエネルギー$\epsilon_{rec}=5\rm{MeV}$を使用する.また,ニュートリノ加熱の影響以外にも,結合エネルギー$\epsilon_{bind}$と核燃焼エネルギー$\epsilon_{burn}$についても考えたのが第2項目である.結合エネルギー$\epsilon_{bind}$については,計算の際にデータとして持ってきて計算を行う.

核燃焼エネルギー$\epsilon_{burn}$については,以下のフラッシングメソッドを用いて計算を行う.


\subsection{フラッシングメソッド}

この「フラッシングメソッド」では,ある一定の温度に達したら核燃焼が一瞬のうちに起こると仮定するもので,衝撃波の温度によって燃焼の段階が変わるというものである.衝撃波の温度については,降着物質の速度が衝撃波の速度より無視できるほど小さいと仮定すると,
$$
P_{sh}=\frac{aT^4_{sh}}{3}=\frac{\beta_{expl}-1}{\beta_{expl}}\rho v^2_{sh}\eqno{(2.33)}
$$
より,次のように求まる.
$$
T_{sh}=\sqrt[4]{\frac{3(\beta_{expl}-1)}{\beta_{expl}}\rho v^2_{sh}}\eqno{(2.34)}
$$
この衝撃波の温度$T_{sh}$によって以下のように組成が変化するとする.
\begin{enumerate}
  \item $2.5\times10^{9}\rm{K}\leq T_{sh}<3.5\times10^{9}$の場合,Oよりも軽い元素が$^{16}$Oまで燃焼する
  \item $3.5\times10^{9}\rm{K}\leq T_{sh}<5.0\times10^{9}$の場合,Siよりも軽い元素が$^{28}$Siまで燃焼する
  \item $5.0\times10^{9}\rm{K}\leq T_{sh}<T_{\alpha}$の場合,Niよりも軽い元素が$^{56}$Niまで燃焼する
\end{enumerate}
この$T_{\alpha}$とは$\alpha$粒子の質量関数が0.5になる時の密度に依存する温度を表し,次の式で与えられる.
$$
\log_{10}{\rho}=11.62+1.5\log_{10}{\left(\frac{T_{\alpha}}{10^9\rm{K}}\right)}-39.17\left(\frac{T_{\alpha}}{10^9\rm{K}}\right)^{-1}\eqno{(2.35)}
$$
計算上では,これをニュートン法をもちいて計算する.そして再帰的に以下の核燃焼エネルギー$\epsilon_{burn}$は,
$$
\epsilon_{burn}=\sum_i X_i\left(\frac{m_i}{A_i}-\frac{m_{end}}{A_{end}}\right)\frac{c^2}{mu}\eqno{(2.36)}
$$
と計算される.ここで,$X_i$は元素$i$の質量分立,$A_i$は元素$i$の質量数,添字$end$は燃焼後に最終的に変化する元素のことを表す.

\subsection{衝撃波速度}

フラッシングメソッドによって,核燃焼エネルギーが求まると爆発のエネルギーを計算でき,最終的に衝撃波の速度を求めることができる.
$$
v_{sh}=0.794\left(\frac{E_{diag}}{M-M_{ini}}\right)^{1/2}\left(\frac{M-M_{ini}}{\rho r^3}\right)^{0.19}\eqno{(2.37)}
$$
ここで出てくる$M_{ini}$は「爆発前段階」における移行直前の質量のことである.

以上により,式(29)の条件を計算できるようになり,次のフェーズに移行するかどうかを判断することができる.


\section{爆発段階(Phase2)}

この段階では,衝撃波に到達した降着物質が全て外向きに跳ね飛ばされるという状況である.衝撃波がそのまま星の表面に達すると爆発が起きたとわかり,その時の爆発のエネルギーや中性子星の質量を計算する.

\subsection{爆発のエネルギー}

前のフェーズ「爆発段階(Phase1)」で考えた爆発のエネルギーは,衝撃波を通過するものも考えていたのでこのフェーズでは少し違う以下の形となる.
$$
\frac{\rm{d} E_{diag}}{\rm{d} M_{sh}}=\epsilon_{bind}+\epsilon_{burn}\eqno{(2.38)}
$$
これらを星表面まで足し合わせることで最終的な爆発のエネルギー$E_{diag}$を得ることができる.

\subsection{中性子星の質量}

中性子星の質量$M_{NS}$はバリオン中性子星の質量$M_{by}$から求めることができる.降着物質が存在している間は,
$$
\frac{\rm{d} M_{by}}{\rm{d} M_{sh}}=(a-\alpha_{out})\left(1-\frac{\eta_{acc}}{|e_g|}\right)\eqno{(2.39)}
$$
でバリオン中性子星の質量$M_{by}$が増えていく.しかしこのフェーズになると降着は止まっているので,バリオン中性子星の質量$M_{by}$は増えず,結合時の質量変化によって次の式を満たす.
$$
M_{NS}=M_{by}-0.084M\sun\left(\frac{M_{NS}}{M\sun}\right)^2\eqno{(2.40)}
$$
これを解いて,中性子星の質量$M_{NS}$は,
$$
M_{NS}=\frac{-1+\sqrt{1+0.336M_{by}/M_{sun}}}{0.168}M_{sun}\eqno{(2.41)}
$$
と求まる.
