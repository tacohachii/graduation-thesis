\chapter{序論}
\label{chap:introduction}

本研究の背景や本論文の構成を書く.


\section{背景}

太陽の何十倍もの質量のある星は,一生の終わりに,超新星により中性子性を残したり,ブラックホールを形成したりする.しかし,現在,どんな構造の星がそれらの過程のどれを経るのかは明確に分かっていない.今回は個々の星について考え,重力崩壊型超新星爆発を起こすケースにおいて超新星爆発後にどのくらいのエネギーを持っていて,さらに放出される元素が何であるかを調べる.そして,その結果を多くの星が集合をなす銀河系に応用できるようにしていく.


\section{本文書の構成}

第\ref{chap:introduction}章では本卒論の概要みたいなのを書いた.第\ref{chap:supernova}章では、主に今回の計算で必要な基本的な現状についてであり,重力崩壊後,爆発を起こすまでの過程についてを説明する.第\ref{chap:mesa_method}章と第\ref{chap:muller_calc}章では,主に計算に用いたコードについてであり,第\ref{chap:mesa_method}章ではMESAを用いた計算の手法について,第\ref{chap:muller_calc}章では重力崩壊後から爆発に到るまでの数値計算の方法についてを説明する.
