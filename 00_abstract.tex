% アブストラクト
\begin{jabstract}

太陽の何十倍もの質量のある星は,一生の終わりに,超新星により中性子性を残したり,ブラックホールを形成したりする.しかし,現在,どんな構造の星がそれらの過程のどれを経るのかは明確に分かっていない.そこで,実際に星の最後がどうなるのかを調べるため,爆発時の過程をコードを作成し計算を行った.計算は前半と後半で分かれており,前半部分は爆発直前の親星の構成を計算し,後半部分は重力崩壊を始めてから衝撃波が星の表面まで到達するまでの過程を追い,最終的に爆発時の爆発のエネルギーを求めた.手法としては,MESA\cite{mesa}で前半部分を,m\"{u}llerの論文\cite{muller}を元にした自作のコードで後半部分をという手法で行った.

\end{jabstract}
